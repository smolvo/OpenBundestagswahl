\documentclass[10pt,a4paper]{article}
\usepackage[utf8]{inputenc}
\usepackage[german]{babel}
\usepackage{amsmath}
\usepackage{amsfonts}
\usepackage{amssymb}

\title{Software zur Mandatsverteilung im Deutschen Bundestag}
\author{Anton Mehlmann, Manuel Olk Enes Ördek, Simon Schürg, Nick Vlasoff, Philipp..}

\begin{document}
\maketitle
\tableofcontents

\section{Produktübersicht}
Bei dem Produkt handelt es sich um ein Programm, das die Sitzverteilung im Deutschen Bundestag gemäß der gesetzlichen Bestimmungen exakt berechnet und deren Zustandekommen verständlich darlegt. Erreicht wird dies durch eine minimalistische und intuitive Benutzeroberfläche.

\subsection{Lizenz}
Der Quellcode des Programms wird der Öffentlichkeit frei zur Verfügung gestellt.

\section{Zielbestimmung}
\subsection{Musskriterien}
\begin{itemize}
\item Berechnung der Sitzverteilung nach gesetzlicher Bestimmung
\item Plattformunabhängigkeit
\item Grafische Benutzeroberfläche
\item Vergleich mehrerer Wahlausgänge
\item Speichern/ Laden vom aktuellen Zustand
\end{itemize}

\subsection{Sollkriterien}
\begin{itemize}
\item Modifikation von Parametern (um ein neues Wahlgesetz zu simulieren)
\item Importmöglichkeit von Wahlergebnissen (json,xml)
\item Auffinden paradoxer Wahlausgänge
\end{itemize}


\subsection{Kannkriterien}
\begin{itemize}
\item Speicherung vergangener Bundestagswahlausgänge (vor 1990?)
\item Export als PDF/Bild
\item Hilfe (Benutzerhandbuch)
\item Kartographische Darstellung der Ergebnisse der Wahlkreise
\item Darstellung von Wahlbeteiligung
\end{itemize}


\subsection{Abgrenzungskriterien}
\begin{itemize}
\item Keine Mobile Anwendung oder Web-App
\item keine Nennung von Abgeordneten
\item keine Werbung für Parteien
\end{itemize}


\section{Produkteinsatz}
\subsection{Anwendungsbereiche}
\begin{itemize}
\item Überprüfung der Wahlergebnisse
\end{itemize}


\subsection{Zielgruppen}
\begin{itemize}
\item Menschen, die kritisch gegenüber Wahlergebnissen sind
\item(unabhängige) Medien
\item politisch interessierte
\end{itemize}


\subsection{Betriebsbedingungen}
\begin{itemize}
\item Keine Verbindung zum Internet nötig
\end{itemize}


\section{Produktumgebung}
\subsection{Software}
\begin{itemize}
\item Java Runtime Environment SE 1.7 oder neuer.
\item Betriebssystem z.B. Windows, Linux, Mac OS
\end{itemize}


\subsection{Hardware}
Mindestanforderungen:
\begin{itemize}
\item Bildschirmauflösung: 1024x768
\end{itemize}


\subsection{Orgware}
\begin{itemize}
\item Keine weiteren Rahmenbedingungen notwendig
\end{itemize}


\subsection{Schnittstellen}
\begin{itemize}
\item Importieren/ Exportieren von Zuständen und Daten
\end{itemize}


\section{Funktionale Anforderungen}
Funktionale Anforderungen werden durch eine vierstellige Nummer gekennzeichnet. Die erste Nummer kennzeichnet den folgenden Bereich:
\begin{itemize}
	\item GUI (View)
	\item Schnittstellen (Controller)
	\item Datenhaltung (Model)
\end{itemize}
Die restlichen Nummern dienen zur Durchnummerierung.

\subsection{GUI (View)}
\begin{itemize}
	\item /F10010/ Programmstart \hfill \\
	Initialisierung des Hauptfensters. Es wird, falls vorhanden, der letzte Programmzustand aufgerufen (/F20010/). Andernfalls lädt das Programm die Daten der letzten Bundestagswahl.
	\item /F10020/ Menü \hfill \\
	Im Menü sind folgende Punkte gelistet:
	\begin{itemize}
		\item Datei
		\begin{itemize}
			\item Speichern des aktuellen Zustandes /F20010/
			\item Laden eines Zustandes /F20011/
			\item Importieren von Daten /F20020/
			\item Exportieren von Daten /F20030/
		\end{itemize}
		\item Bearbeiten
		\begin{itemize}
			\item Rückgängig /F20040/
		\end{itemize}
		\item Extras
		\item Hilfe /F10030/
	\end{itemize}
	\item /F10030/ Hilfe \hfill \\
	Durch den Klick auf diesen Menüpunkt öffnet sich ein Fenster, in der Handbücher zum Programm zu finden sind. Des weiteren ein kleines About, mit den wichtigsten Informationen zum Programm.
	\item /F10050/ Kartografische Darstellung der Länder \hfill \\
	Die Länder werden nach einer überprüfung (/F30010/), Kartografisch im Fenster dargestellt.
	\item /F10100/ Programmende \hfill \\
	Das Programm speichert den aktuellen Zustand (/F20010/) und schließt das Programm.
\end{itemize}

\subsection{Schnittstellen (Controller)}
\begin{itemize}
	\item /F20010/ Speichern des aktuellen Programmzustandes \hfill \\
	Es wird das Zustands-Objekt (/PD01/) in einer Datei abgelegt. Dabei kann der Benutzer beim Speichern einen internen Namen und ein Kommentar abgeben, welche mit gespeichert werden.
	\item /F20011/ Laden eines Programmzustandes \hfill \\
	Der Benutzer wählt eine Datei aus. Es wird überprüft, ob es sich um ein gültiges Objekt handelt. Falls die Version unterschiedlich ist, wird die Datei konvertiert und geladen.
	\item /F20020/ Importieren von Daten \hfill \\
	Die .csv-Dateien des Bundeswahlleiters können importiert werden.
	\item /F20030/ Exportieren von Daten \hfill \\
	Daten können als .csv-Dateien oder als JSON/XML exportiert werden.
	\item /F20040/ \"Rückgängig machen\" \hfill \\
	Es können Operationen rückgängig gemacht werden. Hierfür wird das History-Objekt (/PD05/) verwendet.
\end{itemize}

\subsection{Datenhaltung und Verarbeitung (Model)}
\begin{itemize}
	\item /F30010/ Überprüfen der Ländernamen \hfill \\
	Überprüft ob die eingegebenen Ländernamen in dem Daten-Objekt (/PD03/) korrekt sind. Falls alle Ländernamen gefunden werden, wir die Kartografische Darstellung (/F10050/) aktiviert.
\end{itemize}
\section{Produktdaten}
\begin{itemize}
	\item /PD01/ Zustands-Object \hfill \\
	Repräsentiert den aktuellen Zustand des Programms. Es beinhaltet alle Informationen, um den genauen Zustand des Programms wiederherzustellen. Es handelt sich um eine serialisierbare Klasse.
	\begin{itemize}
		\item Version des Programms (um Abwärtskompatibilität zu gewährleisten)
		\item Datum und Uhrzeit der Erstellung
		\item Name/ID
		\item Kommentar
		\item Fenster-Objekt als Liste
		\item History-Objekt
	\end{itemize}
	
	\item /PD02/ Fenster-Objekt
	\begin{itemize}
		\item Wahlergebnisse (Daten-Objekt)
		\item Wahlgesetz-Objekt
		\item Name/ID
	\end{itemize}
	
	\item /PD03/ Daten-Objekt \hfill \\
	Beinhaltet die Anzahl der (Erst- und Zweit-)Stimmen je Wahlkreis und Partei.
	\begin{itemize}
		\item Name/ID
		\item Kommentar (Quelle)
		\item Wahlkreise mit Stimmen als Array.
	\end{itemize}
	
	\item /PD04/ Wahlgesetz-Objekt \hfill \\
	Beinhaltet das Algorithmus zur Berechnung der Sitze mit dem Daten-Objekt als Eingabedatum.\\
	Des weiteren überprüft das Objekt, ob mit dem Datenobjekt eine Wahl überhaupt simuliert werden kann (wenn ein Datenobjekt beispielsweise keine Erststimmen enthält wie in sehr alten Wahlen).
	
	\item /PD05/ History-Objekt \hfill \\
	Dieses Objekt zeichnet alle Veränderungen am Programm auf. Mithilfe dieses Objektes können Operationen über das Menü Bearbeiten oder STRG+Z rückgängig gemacht werden (/F10020/ und /F20040/).
\end{itemize}
\section{Produktleistungen}
\begin{itemize}
	\item Zeit
	\begin{itemize}
		\item Starten + Laden des letzten Zustandes: unter 5 Sekunden.
		\item Beenden + Speichern des aktuellen Zustandes: unter 5 Sekunden.
		\item Exportieren/Importieren von Daten: unter 10 Sekunden.
	\end{itemize}
	\item Genauigkeit \hfill \\
	Die Genauigkeit des Algorithmus zur Sitzberechnung muss dem Wahlgesetz entsprechen und exakte Ergebnisse liefern.
\end{itemize}
\section{Nicht-funktionale Anforderungen}
\begin{itemize}
	\item Allgemeine Anforderungen:
	\begin{itemize}
		\item Die Sitzverteilung muss für den Benutzer transparent und nachvollziehbar dargestellt werden.
	\end{itemize}
	\item Sicherheitsanforderungen:
	\begin{itemize}
		\item Noch nichts...
	\end{itemize}
	\item Plattformunabhängigkeit: \hfill \\
	Das Programm muss auf der offiziellen Oracle JRE laufen.
\end{itemize}
\section{Qualitätsanforderungen}
\begin{itemize}
	\item Hilfreiche Fehlermeldungen
	\item Kein Datenverlust (auch nach Programmabstürzen)
\end{itemize}
\section{Globale Testfälle und Szenarien}

\section{Systemmodelle}

\section{Benutzungsoberfläche}

\section{Spezielle Anforderungen an die Entwicklungsumgebung}
\begin{itemize}
	\item Allgemein
	\begin{itemize}
		\item Latex
		\item Versionskontrolle mit SVN
	\end{itemize}
	\item Entwicklung
	\begin{itemize}
		\item IDE: Eclipse
	\end{itemize}
	\item Entwurf
	\begin{itemize}
		\item DIA für Diagramme
	\end{itemize}
	\item Validierung
	\begin{itemize}
		\item JUnit
	\end{itemize}
	\item Teamkommunikation
	\begin{itemize}
		\item Google-Groups Mailingliste
	\end{itemize}
\end{itemize}

\section{Zeit- und Ressourcenplanung}

\section{Ergänzungen}

\section{Glossar}



\end{document}
