\documentclass[10pt,a4paper]{article}
\usepackage[utf8]{inputenc}
\usepackage[german]{babel}
\usepackage{amsmath}
\usepackage{amsfonts}
\usepackage{amssymb}

\title{Software zur Mandatsverteilung im Deutschen Bundestag}
\author{Anton Mehlmann, Manuel Olk Enes Ördek, Simon Schürg, Nick Vlasoff, Philipp..}

\begin{document}
\maketitle
\tableofcontents

\section{Produktübersicht}
Bei dem Produkt handelt es sich um ein Programm, das die Sitzverteilung im Deutschen Bundestag gemäß der gesetzlichen Bestimmungen exakt berechnet und deren Zustandekommen verständlich darlegt. Erreicht wird dies durch eine minimalistische und intuitive Benutzeroberfläche.

\subsection{Lizenz}
Der Quellcode des Programms wird der Öffentlichkeit frei zur Verfügung gestellt.

\section{Zielbestimmung}
\subsection{Musskriterien}
\begin{itemize}
\item Berechnung der Sitzverteilung nach gesetzlicher Bestimmung
\item Plattformunabhängigkeit
\item Grafische Benutzeroberfläche
\item Vergleich mehrerer Wahlausgänge
\item Speichern/ Laden vom aktuellen Zustand
\end{itemize}

\subsection{Sollkriterien}
\begin{itemize}
\item Modifikation von Parametern (um ein neues Wahlgesetz zu simulieren)
\item Importmöglichkeit von Wahlergebnissen (json,xml)
\item Auffinden paradoxer Wahlausgänge
\end{itemize}


\subsection{Kannkriterien}
\begin{itemize}
\item Speicherung vergangener Bundestagswahlausgänge (vor 1990?)
\item Export als PDF/Bild
\item Hilfe (Benutzerhandbuch)
\item Kartographische Darstellung der Ergebnisse der Wahlkreise
\item Darstellung von Wahlbeteiligung
\end{itemize}


\subsection{Abgrenzungskriterien}
\begin{itemize}
\item Keine Mobile Anwendung oder Web-App
\item keine Nennung von Abgeordneten
\item keine Werbung für Parteien
\end{itemize}


\section{Produkteinsatz}
\subsection{Anwendungsbereiche}
\begin{itemize}
\item Überprüfung der Wahlergebnisse
\end{itemize}


\subsection{Zielgruppen}
\begin{itemize}
\item Menschen, die kritisch gegenüber Wahlergebnissen sind
\item(unabhängige) Medien
\item politisch interessierte
\end{itemize}


\subsection{Betriebsbedingungen}
\begin{itemize}
\item Keine Verbindung zum Internet nötig
\end{itemize}


\section{Produktumgebung}
\subsection{Software}
\begin{itemize}
\item Java Runtime Environment SE 1.7 oder neuer.
\item Betriebssystem z.B. Windows, Linux, Mac OS
\end{itemize}


\subsection{Hardware}
Mindestanforderungen:
\begin{itemize}
\item Bildschirmauflösung: 1024x768
\end{itemize}


\subsection{Orgware}
\begin{itemize}
\item Keine weiteren Rahmenbedingungen notwendig
\end{itemize}


\subsection{Schnittstellen}
\begin{itemize}
\item Importieren/ Exportieren von Zuständen und Daten
\end{itemize}


\section{Funktionale Anforderungen}


\section{Globale Testfälle und Testszenarien}

\section{Systemmodelle}

\section{Benutzungsoberfläche}

\section{Spezielle Anforderungen an die Entwicklungsumgebung}

\section{Zeit- und Ressourcenplanung}

\section{Ergänzungen}

\section{Glossar}



\end{document}
