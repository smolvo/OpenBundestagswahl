\documentclass[10pt,a4paper]{article}
\usepackage[utf8]{inputenc}
\usepackage[german]{babel}
\usepackage{amsmath}
\usepackage{amsfonts}
\usepackage{amssymb}
\usepackage{setspace}
\usepackage{graphicx} %Um Bilder anzeigen zu können
\usepackage[top=1in, bottom=1.5in, left=1in, right=1in]{geometry}

\title{\Huge Pflichtenheft\\[1cm] {\bfseries Praxis der Softwareentwicklung}\\Gruppe 1\\[2cm] Entwicklung einer Software zur Berechnung der Mandatsverteilung im Deutschen Bundestag\\[1cm] }
\author{Philipp Löwer, Anton Mehlmann, Manuel Olk, Enes Ördek, \\Simon Schürg, Nick Vlasoff}
\date{}

\begin{document}
\maketitle

\begin{figure}[h]

\centering
		
		\includegraphics[scale=0.6]{KIT-Logo.png}\\
		\Huge WS 2013 / 14
\end{figure}

		
\newpage
\begin{onehalfspace}
\tableofcontents
\end{onehalfspace}
\newpage 

\section{Produktübersicht}
Das Produkt soll allen Personen, auch ohne spezifisches Vorwissen, die sich mit der Deutschen Bundestagswahl beschäftigen, als Unterstützung dienen.\newline
Dafür ist die Aufbereitung von Wahlergebnissen gemäß der gesetzlichen Bestimmungen und deren exakte und übersichtliche Darstellung, z.B. der endgültigen Sitzverteilung im Deutschen Bundestag, essentiell.
Da das aktuelle Wahlsystem sehr komplex ist, besteht eine weitere Aufgabe des Programms darin, das Zustandekommen von Direktmandaten, Überhangmandaten, Ausgleichsmandaten usw. dem Anwender verständlich zu erklären.
Darunter fallen auch paradox erscheinende Vorkommnisse, wie das negative Stimmgewicht, die durch das Programm gefunden und erklärt werden sollen. \newline
Des Weiteren ermöglicht die Anwendung mit beliebigen Wahldaten zu experimentieren und die daraus resultierenden Veränderungen anzuzeigen. Dadurch lässt es sich auch gut für Demonstrationen (z.B. für das Aufzeigen von Rundungsfehlern bei der Sitzberechnung) nutzen.


\subsection{Lizenz}
Es gibt bereits kommerzielle Programme, die dem Produkt ähneln. Diese sind aber nicht frei verfügbar und weisen meistens gerade in Bezug auf die Nutzerfreundlichkeit für Laien erhebliche Mängel auf. Genau dort wird nun Abhilfe geschafft.\newline
Der Quellcode wird der Öffentlichkeit frei zur Verfügung gestellt, um Interessenten die Bundestagswahl und ihre Besonderheiten kostenlos und nachvollziehbar näher zu bringen. Es wird die GPL V3 Lizenz verwendet, damit das Projekt beliebig erweitert bzw. modifiziert werden kann und trotzdem immer noch als freie Software gilt.



\section{Zielbestimmung}
\subsection{Musskriterien}
\begin{itemize}
\item Auswertung von Wahlergebnissen nach gesetzlicher Bestimmung
\begin{itemize}
\item Berechnen der Direkt-, Überhang- und Ausgleichsmandate
\item Berechnen der restlichen Sitzverteilung im Deutschen Bundestag anhand der Zweitstimmen
\end{itemize}
\item Anzeigen und Interaktion mit Wahlausgängen anhand einer grafischen Benutzeroberfläche
\item Direkte Gegenüberstellung verschiedener Wahlausgänge (z.B. Wahlausgang 2013 - Wahlausgang 2009)
\item Importmöglichkeit von Wahlergebnissen (.csv)
\item Manuelle Änderung von importierten Wahlergebnissen (z.B. Zweitstimmen für eine Partei erhöhen)
\end{itemize}

\subsection{Sollkriterien}
\begin{itemize}
\item Auffinden von Wahlausgängen zu paradoxen Vorkommnissen (z.B. negatives Stimmgewicht)
\item Kartografische Darstellung der Bundesländer
\end{itemize}

\subsection{Kannkriterien}
\begin{itemize}
\item Hilfe (Benutzerhandbuch)
\item Anzeigen von Koalitionsmöglichkeiten
\end{itemize}


\subsection{Abgrenzungskriterien}
\begin{itemize}
\item Keine mobile Anwendung oder Web- Applikation
\item Keine Übersetzung in andere Sprachen
\item Keine namentliche Nennung von Abgeordneten
\end{itemize}


\section{Produkteinsatz}
\subsection{Anwendungsbereiche}
\begin{itemize}
\item Kann genutzt werden, ...
\begin{itemize}
\item um Wahlausgänge zu simulieren
\item um komplexes Wahlsystem nachzuvollziehen 
\item um bestimmte Sachverhalte (z.B. negatives Stimmrecht, Benachteiligung kleiner Parteien) darzustellen 
\end{itemize} 
\end{itemize}


\subsection{Zielgruppen}
\begin{itemize}
\item Politisch Interessierte
\item Medien (z.B. Internetseiten, regionale Zeitungen)
\end{itemize}


\subsection{Betriebsbedingungen}
\begin{itemize}
\item Die Verbindung zum Internet ist optional
\end{itemize}


\section{Produktumgebung}
\subsection{Software}
\begin{itemize}
\item Java Runtime Environment SE 1.7 oder neuer.
\item Betriebssystem z.B. Windows, Linux, Mac OS
\end{itemize}


\subsection{Hardware}
Das Programm ist für den Einsatz auf PCs oder Laptops geeignet.
\subparagraph{Mindestanforderungen:}
\begin{itemize}
\item 128 MB Arbeitsspeicher
\item 100 MB freien Festplattenspeicher
\item 500-MHz-Prozessor
\item Farbdisplay/ Bildschirmauflösung: 1024x768
\end{itemize}

\subparagraph{Empfohlen:}
\begin{itemize}
\item 512 MB Arbeitsspeicher
\item 100 MB freien Festplattenspeicher
\item 1-GHz-Prozessor
\item Farbdisplay/ Bildschirmauflösung: 1024x768
\end{itemize}


\subsection{Orgware}
\begin{itemize}
\item Keine weiteren Rahmenbedingungen notwendig
\end{itemize}


\subsection{Schnittstellen}
\begin{itemize}
\item Importieren/Exportieren von Wahldaten (.csv)
\end{itemize}


\section{Funktionale Anforderungen}
Funktionale Anforderungen werden durch eine vierstellige Nummer gekennzeichnet. Die erste Nummer kennzeichnet den folgenden Bereich:
\begin{enumerate}
	\item GUI
	\item Schnittstellen
	\item Datenhaltung
\end{enumerate}
Die restlichen Nummern dienen zur Durchnummerierung.

\newpage
\subsection{GUI}
\begin{itemize}
	\item /F10010/ Programmstart \hfill \\
	Es erscheint ein Startfenster. Der Benutzer hat die Wahl zwischen dem Laden eines vorher gespeicherten Programmzustandes, falls vorhanden dem letzten Programmzustand (/F20010/) oder dem Importieren von Wahlergebnissen.
	\item /F10020/ Menü \hfill \\
	Im Menü sind folgende Punkte gelistet:
	\begin{itemize}
		\item Datei
		\begin{itemize}
			\item Neuer Tab /F20001/
			\item Tab schließen /F20002/
			\item Speichern des aktuellen Zustandes /F20010/
			\item Laden eines Zustandes /F20011/
			\item Importieren von Daten /F20020/
			\item Exportieren von Daten /F20030/
			\item Beenden /F20070/
		\end{itemize}
		\item Bearbeiten
		\begin{itemize}
			\item Rückgängig /F20040/
			\item Wiederherstellen /F20041/
		\end{itemize}
		\item Extras
		\begin{itemize}
			\item Ändern der aktuellen Ansicht /F10091/
			\item Wahlgesetz auswählen /F10092/
			\item Vergleichen mit... /F10093/
		\end{itemize}
		\item Hilfe /F10030/
	\end{itemize}
	\item /F20091/ Neuer Tab \hfill \\
	Beim Anklicken wird ein neuer Tab (/PD02/) generiert, in dem eine neue Bundestagswahl (/PD03/) geladen werden kann. Diese Dateiauswahl korrespondiert zu dem '+'-Button in der Tab-Leiste.
	\item /F20092/ Tab schließen \hfill \\
	Beim Anklicken wird der aktuelle Tab (/PD02/) geschlossen. Wurden Änderung an den Dateien (/PD02/) vorgenommen, wird vor dem Schließen des Tabs dem Benutzer die Möglichkeit gegeben seine Einstellungen zu speichern, zu verwerfen oder das Schließen abzubrechen. Diese Dateiauswahl korrespondiert zu dem 'x'-Button in jedem Tab.
	\item /F10091/ Beenden \hfill \\
	Beendet das gesamte Programm. Wurden Änderung an den Dateien (/PD02/) vorgenommen, wird vor dem Schließen des Tabs dem Benutzer die Möglichkeit gegeben seine Einstellungen zu speichern, zu verwerfen 
	oder das Schließen abzubrechen.
	\item /F20093/ Rückgängig machen \hfill \\
	Wurde eine Stimmzahl, ob in einem Wahlkreis oder in einem ganzen Bundesland, verändert, wird der vorhergehende Wert wieder hergestellt.
	\item /F20094/ Wiederherstellen \hfill \\
	Falls Änderungen rückgängig gemacht wurden (/F20093/), können diese wieder hergestellt werden.
	\item /F10030/ Hilfe \hfill \\
	Durch den Klick auf diesen Menüpunkt öffnet sich ein Fenster, in der Handbücher zum Programm zu finden sind. Des weiteren ist ein kleines 'About', mit den wichtigsten Informationen zum Programm, vorhanden.
	\item /F10060/ Programmfenster \hfill \\
	Das Programmfenster wird in drei Bereiche aufgeteilt.
	\begin{itemize}
		\item Tabellenfenster /F10070/
		\item Diagrammfenster /F10080/
		\item Kartenfenster /F10090/
	\end{itemize}
	Es gibt hierbei zwei verschiedene Ansichten.
	\begin{itemize}
		\item Bundesansicht /F10094/
		\item Landesansicht /F10095/
	\end{itemize}
	\item /F10070/ Tabellenfenster \hfill \\
	Es werden Bundesländer in tabellarischer Form angezeigt. Klickt der Benutzer auf ein Bundesland, werden die Wahlkreise des betroffenen Bundeslandes angezeigt. Der Klick auf die Wahlkreise öffnet die Ergebnisse für die Parteien, die Anzahl der Wahlberechtigten und die Zweitstimmen. Im Kartenfenster (/F10090/) wird die Deutschlandkarte durch ein Bild des angeklickten Bundeslandes ersetzt und im Diagrammfenster /(/F10080/) erscheint ein Diagramm zu den Zweitstimmenergebnissen aller Parteien. \\
	Es gibt die Möglichkeit die Sortierung der Tabelle zu ändern. (/F10110/) \\
	Ein Zurück-Pfeil wechselt von Bundesland- zur Deutschlandansicht. (/F10120/)
	\item /F10080/ Diagrammfenster \hfill \\
	Im Diagrammfenster sieht man Diagramme den Daten (/PD03/) entsprechend. Befindet man sich in der Bundesansicht (/F10094/) wird die Sitzplatzverteilung (/F10100/) für alle Parteien, die es in den Bundestag geschafft haben, angezeigt. Wurde ein bestimmtes Bundesland vom Benutzer gewählt, zeigt das Diagramm die prozentuale Anzahl der Zweitstimmen, die die einzelnen Parteien bekommen haben.
	\item /F10090/ Kartenfenster \hfill \\
	Die Länder werden nach einer Überprüfung (/F30010/), ob die Namen übereinstimmen, kartografisch im Fenster dargestellt und nach den Parteien, die die meisten Stimmen in diesem Bundesland erhielten eingefärbt. Der Klick auf ein Bundesland öffnet die Landesansicht (/F10095/). \\
	Ein Zurück-Pfeil wechselt zur vorherigen Bundesansicht (/F10094/).
	\item /F10091/ Ändern der aktuellen Ansicht \hfill \\
	\item /F10092/ Wahlgesetz auswählen \hfill \\
	Hier kann das Wahlgesetz, welches für die Auswertung der Daten (/PD03/) und die Sitzplatzverteilung im Bundestag verwendet werden soll ausgewählt werden. Mitgeliefert wird das aktuelle Wahlgesetz (2013) und das Vorhergehende (2009).
	\item /F10093/ Vergleichen mit... \hfill \\
	Mit diesem Feature soll es möglich sein den Ausgang der Bundestagswahl des aktuellen Tabs mit einer anderen geladenen Bundestagswahl zu vergleichen. Ist keine andere Wahl gerade geöffnet, wird dem Benutzer erst empfohlen eine weitere Datei (/PD03/) in einen neuen Tab zu laden.
	\item /F10094/ Bundesansicht \hfill \\
	In dieser Ansicht wird das gesamte Bundesland betrachtet. Im Kartenfenster (/F10090) sieht man die gefärbte Deutschlandkarte, im Tabellenfenster (/F10070/) werden alle Bundesländer aufgelistet und im Diagrammfenster (/F10080/) sieht man ein Diagramm über die Sitzplatzverteilung im deutschen Bundestag.
	\item /F10095/ Landesansicht \hfill \\
	In dieser Ansicht wird ein ausgewähltes Bundesland betrachtet. Im Kartenfenster (/F10090) sieht man ein kartografisches Bild des Bundeslandes, im Tabellenfenster (/F10070/) werden z.B. Wahlbeteiligung, Anzahl der Zweitstimmen angezeigt, und alle Wahlkreise, die zu diesem Bundesland gehören. Im Diagrammfenster (/F10080/) sieht man ein Diagramm über die prozentuale Zweitstimmenanzahl der einzelnen Parteien.
	\item F10100 Sitzverteilung \hfill \\
	Die Sitzverteilung wird dargestellt mit einem Kuchendiagramm. Daneben kann man sich auch anzeigen lassen, wie jeder einzelne Sitz entstanden ist. Dies wird tabellarisch in einem neuen Fenster angezeigt.
	\item F10110 Sortierung des Tabellenfensters \hfill \\
	Die Sortierung des Tabellenfensters kann mithilfe eines Dropdown-Menüs geändert werden. Folgende Sortierungen sind möglich:
	\begin{itemize}
		\item Bundesländer (Standard)
		\item Parteien
	\end{itemize}
	\item F10120 "Zurück"-Button bei Ansichten.
	Der Klick auf diesen Button wechselt von der Landesansicht (/F10095/) in die Bundesansicht (/F10094/).
\end{itemize}

\subsection{Schnittstellen}
\begin{itemize}
	\item /F20010/ Speichern des aktuellen Programmzustandes \hfill \\
	Es wird das Zustands-Objekt (/PD01/) in einer Datei abgelegt. Dabei kann der Benutzer beim Speichern einen internen Namen und ein Kommentar abgeben, welche mit gespeichert werden.
	\item /F20011/ Laden eines Programmzustandes \hfill \\
	Der Benutzer wählt eine Datei (???) aus. Es wird überprüft, ob es sich um ein gültiges Objekt handelt. Falls die Version unterschiedlich ist, wird eine Fehlermeldung ausgegeben, die dem Benutzer empfiehlt eine Datei der aktuellen Version auszuwählen.
	\item /F20020/ Importieren von Daten \hfill \\
	Die .csv-Dateien des Bundeswahlleiters können importiert werden.
	\item /F20030/ Exportieren von Daten \hfill \\
	Daten können als .csv-Dateien exportiert werden.
	

\end{itemize}

\subsection{Datenhaltung und Verarbeitung}
\begin{itemize}
	\item /F30010/ Überprüfen der Ländernamen \hfill \\
	Überprüft, ob die eingegebenen Ländernamen in dem Daten-Objekt (/PD03/) korrekt sind. Falls alle Ländernamen gefunden werden, wir die kartografische Darstellung (/F10090/) aktiviert.
	\item /F30020/ Überprüfen der Stimmen\hfill \\
	Überprüft ob die eingegebenen Stimmen in dem Daten-Objekt (/PD03/) korrekt sind. Falls die Anzahl der Stimmen $\geq$ 0 sind, kann die Sitzverteilung mit dem Wahlgesetz-Objekt (/PD04/) berechnet werden.
	\item /F30030/ Filtern der relevanten Daten \hfill \\
	Die benötigten Dateien, wie z.B. Erststimmengewinner der einzelnen Wahlkreise, Zweitstimmen aller Wahlkreise, werden aus der ausgewählten .csv-Datei geladen und zu der Model-Klasse geschickt.			
	\item /F30040/ Generierung von Wahldaten \hfill \\
	???
	\item /F30050/ Manuelles ändern einzelner Stimmzahlen. \hfill \\
	Der Benutzer kann in dem Tabellenfenster (/F10070/) die Zahlen der aktuellen Wahlsimulation manuell anpassen.
	Die hat sofortigen Einfluss auf Karten- (/F10090/) und Diagrammfenster (/F10080/).
	\item /F30060/ Paradoxe Wahlausgänge \hfill \\
	Die aktuell geladenen Wahlausgänge (/PD02/) werden auf paradoxe Eigenschaften überprüft.
	\item /F30070/ Auswerten der Wahlergebnisse \hfill \\
	Nachdem die Wahlergebnisse (/PD02/) entweder geladen oder verändert wurden, werden sie ausgewertet, d.h. es werden die Diagramme erstellt. Nachdem dies erfolgreich vollführt wurde, wird die Deutschlandkarte gefärbt (/F30091/). 
	\item /F30080/ Verändern der Wahlergebnisse \hfill \\
	In dem Tabellenfenster (/F10070/) können Stimmenanzahlen einzelner Wahlkreise verändert werden, woraufhin gleich die Diagramme aktualisiert werden.
	\item /F30090/ Generierung zufälliger Wahlergebnisse \hfill \\
	Es ist möglich einen Wahlausgang zufällig simulieren zu lassen, wobei berücksichtigt wird, dass dabei kein paradoxer Wahlgang (/F30060/) herauskommt.
	\item /F30091/ Färben der Bundesländer \hfill \\
	Wurden Bundeslandsnamen (/F30010/) und Stimmen (/F30020/) überprüft werden die einzelnen Bundesländer mit der Farbe der Partei eingefärbt, die die meisten Zweitstimmen in diesem Bundesland erhalten hat.
	
\end{itemize}

\section{Produktdaten}
\begin{itemize}
	\item /PD01/ Programmzustand
	\begin{itemize}
		\item Version des Programms
		\item Name
	\end{itemize}
	
	\item /PD02/ Wahlergebnis
	\begin{itemize}
		\item Name der Wahl
		\item Kommentar (Quelle)
		\item Wahlkreis/Bundesland mit Stimmen je Partei
		\item Anzahl der Wahlberechtigten
		\item Wahlergebnisse 2009 und 2013
		\item Anzahl der (Erst- und Zweit-)Stimmen je Wahlkreis/Bundesland und Partei
	\end{itemize}
	
	\item /PD03/ Parteien
	\begin{itemize}
		\item Farbe
		\item Vollständiger Name
		\item Kürzel
	\end{itemize}
	
	\item /PD04/ Bundesland
	\begin{itemize}
		\item Wappen		
		\item Vollständiger Name
		\item Kürzel
	\end{itemize}
	
	\item /PD05/ Handbuch
	\begin{itemize}
		\item Informationen zum Programm (About)
		\begin{itemize}
			\item Autoren
			\item Webseite
		\end{itemize}
		\item Informationen zum Wahlsystem
	\end{itemize}
\end{itemize}

\section{Produktleistungen}
\begin{itemize}
	\item Zeit \hfill \\
	Das Programm muss fähig sein, Operationen der letzten zwei Bundestagswahlen, in angemessener Zeit durchzuführen. Entscheidend sind hierbei die Anzahl der Parteien und die Anzahl der Wahlkreise.
	\\ Wir nehmen daher folgende Bedingungen an:
	\begin{itemize}
		\item etwa 30 Parteien
		\item etwa 300 Wahlkreise
		\item maximal 100.000.000 Wahlberechtigte
	\end{itemize}
	Folgende Zeiten werden benötigt:
	\begin{itemize}
		\item Starten des Programms: unter 10 Sekunden
		\item Laden eines Zustandes: unter 10 Sekunden
		\item Berechnung der Sitzverteilung: unter 10 Sekunden
		\item Speichern eines Zustandes: unter 10 Sekunden
		\item Exportieren/Importieren von Daten: unter 10 Sekunden
		\item Beenden des Programms: unter 3 Sekunden
	\end{itemize}
	\item Genauigkeit \hfill \\
	Die Genauigkeit des Algorithmus zur Sitzberechnung muss dem Wahlgesetz entsprechen und exakte Ergebnisse liefern.
\end{itemize}


\section{Nicht-funktionale Anforderungen}
\subsection{Allgemeine Anforderungen}
Die Sitzverteilung muss für den Benutzer möglichst nachvollziehbar dargestellt werden. \hfill \\
Dies wird erreicht durch die folgenden Funktionen:
\begin{itemize}
	\item /F10080/: Ansicht der Sitzplatzverteilung in Diagramm-Form
	\item /F10100/: Der Benutzer kann verfolgen, wie ein Sitz entstanden ist
\end{itemize}

\subsection{Sicherheitsanforderungen}
Die Eingabedaten dürfen während der Berechnung nicht verändert werden.

\subsection{Plattformunabhängigkeit}
Das Programm muss mit der offiziellen Oracle JRE laufen.

\subsection{Erweiterbarkeit}
Da es sich um ein Open-Source Projekt handelt, wird es so entwickelt, dass das Programm gut erweiterbar ist.
	
\section{Qualitätsanforderungen}
\begin{itemize}
	\item Hilfreiche Fehlermeldungen
	\item Kein Datenverlust (auch nach Programmabstürzen)
	\item Gespeicherte Daten müssen immer konsistent gehalten werden
	\item Kurze Einarbeitungszeit
	\item Keine Verklemmungen
\end{itemize}
\newpage
\section{Globale Testfälle und Szenarien}
Folgende Funktionssequenzen sind zu überprüfen:

\includegraphics[scale=0.4]{Diagramm2}
\newpage
Folgende Datenkonsistenzen müssen eingehalten werden:
\begin{itemize}
	\item Zustände können nur gespeichert werden, wenn alle geladenen Felder ausgefüllt wurden
\end{itemize}
Folgende unzulässige Aktionen müssen korrekt behandelt werden:
\begin{itemize}
	\item Negative Stimmenanzahl
	\item Buchstaben als Stimmen
\end{itemize}
Testszenarien:
Die folgenden Testfälle testen das Import-/Exportverhalten des Programms. Dabei wird vorausgestzt, dass das Programm gestartet wurde und sich im Startzustand befindet. 
\begin{itemize}
	\item /T0010/ Struktur einer .csv-Datei verändern: \newline
	Verändern der .csv-Datei $\rightarrow$ Im Hauptmenü auf Datei klicken $\rightarrow$ Datei importieren auswählen $\rightarrow$ Im Dateibrowser die korrupte .csv-Datei auswählen $\rightarrow$ Mit dem Button Laden bestätigen $\rightarrow$ Fehlermeldung: "Datei konnte nicht geladen werden" $\rightarrow$ Fehlermeldung bestätigen
	\item /T0020/ Zum gespeicherten Zustand zurückkehren: \\
	Beliebige .csv-Daten laden $\rightarrow$ Die Sitzverteilung berechnen lassen$\rightarrow$ Aktuellen Zustand speichern als .csv-Datei $\rightarrow$ Beliebige Veränderungen an den Stimmen ausführen $\rightarrow$ Programm ohne zu speichern schließen $\rightarrow$ Programm wieder öffnen und den Zustand ohne die Veränderungen laden
	\item /T0030/ Gleichnamige Parteien in der .csv-Datei: \newline
	.csv-Datei mit zwei gleichnamigen Parteien laden $\rightarrow$ Fehlermeldung: Mehrere Parteien haben den gleichen Namen 
	\item /T0040/ Nur eine Partei in der .csv-Datei: \newline
	Beliebige .csv-Datei mit nur einer Partei laden $\rightarrow$ Sitzverteilung berechnen lassen $\rightarrow$ Fehlermeldung: Sitzverteilung mit nur einer Partei nicht möglich 
	\item /T0050/ Kartografische Ansicht nicht möglich: \newline
	.csv-Datei mit mindestens einem falsch geschriebenen $\rightarrow$ Fehlermeldung: "Kartografische Ansicht nicht möglich" $\rightarrow$ Sitzverteilung der restlichen mit Hilfe der restlichen Bundesländer berechnen\newline
\end{itemize}
Die folgenden Testfälle testen die korrekte Berechnung der Sitzverteilung. Dabei wird vorausgesetzt, dass  das Programm gestartet wurde und erfolgreich eine .csv-Datei geladen wurde.
\begin{itemize}
	\item /T0110/ Direktmandat fehlt: \newline
	Manuell alle Erststimmen eines Wahlkreises löschen $\rightarrow$ Berechnung wird gestartet $\rightarrow$ Erkennen des fehlenden Direktmandats $\rightarrow$ Fehlermeldung: Direktmandat fehlt im Wahlkreis $\rightarrow$ Berechnung abbrechen
	\item /T0120/ Mehrere Wahlkandidaten haben gleich viele Stimme in einem Wahlkreis: \newline
	Mehrere Kanditaten haben gleich viele Stimmen $\rightarrow$ Hinweis wird ausgegeben $\rightarrow$ Kanditat wird ausgelost
	\item /T0130/ Negative Stimmgewicht provozieren:\newline
	Daten so modifizieren, dass ein negatives Stimmgewicht provoziert wird $\rightarrow$ Hinweis: Negatives Stimmgewicht  
	\item /T0140/ Partei mit drei Direktmandate und 2.9 Prozent der Zweitstimmen: \newline
	Eine Partei hat genau drei Direktmandate $\rightarrow$ Sitzverteilung berechnen $\rightarrow$ Partei ist im Bundestag
	\item /T0150/ Überhangmandat testen: \newline
	Eine Partei hat mehr Direktmandate als durch Zweitstimmen zugeteilte Sitze $\rightarrow$ Sitzverteilung  berechnen $\rightarrow$ Sitzverteilung-Liste aufrufen $\rightarrow$ Eintrag mit dem 	Überhangmandaten finden
	\item /T0160/ Ausgleichsmandat testen: \newline
	$\rightarrow$ Sitzverteilung-Liste	aufrufen $\rightarrow$ Eintrag mit dem 
	Ausgleichsmandat überprüfen
\end{itemize}
Im Anschluss werden noch grundlegende Funktionen des Programms getestet, um sicherzustellen, dass das Programm während der Bearbeitung nicht durch eine fehlerhafte Aktion abstürtzt.
\begin{itemize}
	\item /T0210/ Zwei Wahlen miteinander vergleichen:
	Sitzverteilung von der geladen Datei berechnen $\rightarrow$ Neuen Tab öffnen $\rightarrow$ Neue .csv-Datei laden $\rightarrow$ Sitzverteilung für die neue Wahl berechnen $\rightarrow$ Beide Sitzverteilung mit einander vergleichen
	\item /T0220/ Zustand exportieren und danach neu laden
	Aktuellen Zustand des Programms speichern $\rightarrow$ Zustand exportieren $\rightarrow$ Auf einem anderen Gerät das Programm starten $\rightarrow$ erstellten Zustand laden
	\item /T0230/ Manuell einen negativen Wert als Stimmenanzahl eintragen
	Negativen Wert als Stimmenanzahl in die Tabelle eintragen $\rightarrow$ Fehlermeldung: "Negativer Wert" $\rightarrow$ Wert unverändert lassen
	\item /T0240/ Manuell Buchstaben als negativen Wert eintragen
	Buchstaben als Stimmenanzahl in die Tabelle eintragen $\rightarrow$ Fehlermeldung: "Buchstaben anstatt einer Zahl eingegeben"
	\item /T0250/ Sitzverteilung mit zwei verschiedenen Wahlgesetzen berechnen
	Sitzverteilung berechnen $\rightarrow$ Ein anderes Wahlgesetz auswählen $\rightarrow$ Erneut die Sitzverteilung berechnen
	\item /T0260/ Diagramm wechseln
	Sirtzverteilung berechnen $\rightarrow$ Diagramm ändern $\rightarrow$ Sitzverteilung wird beibehalten
	\item /T0270/ Rückgängig machen
	Stimmenanzahl verändern $\rightarrow$ Sitzverteilung berechnen $\rightarrow$ Durch Rückgängig machen den Wert wieder zurücksetzen $\rightarrow$ Sitzverteilung berechnen
\end{itemize}
\section{Systemmodelle}
\subsection{Systemarchitektur}
Das Programm basiert auf der MVC- Architektur, wobei auf eine saubere Trennung der Einheiten Model, View und Controller geachtet wird. Dies sorgt nicht nur für einen flexiblen Programmentwurf, so dass spätere Änderungen bzw. Erweiterungen erleichtert werden, sondern garantiert auch die Trennung kritischer Komponenten, wie der Algorithmusimplementierung, von weniger sensiblen Komponenten, wie der GUI, und dient allgemein der Übersichtlichkeit.

\section{Benutzungsoberfläche}

\section{Spezielle Anforderungen an die Entwicklungsumgebung}
\begin{itemize}
	\item Allgemein
	\begin{itemize}
		\item \textbf{$\LaTeX$} - zur Erstellung von Dokumenten
		\item \textbf{Subversion} (SVN) - zur Versionsverwaltung
		\item \textbf{EtherPad} - zur kollaborativen Bearbeitung von Texten
	\end{itemize}
	\item Entwicklung
	\begin{itemize}
		\item \textbf{Eclipse} - integrierte Entwicklungsumgebung (IDE)
		\item \textbf{Swing} - zur Erstellung der grafischen Benutzeroberfläche
	\end{itemize}
	\item UML und Diagramme
	\begin{itemize}
		\item \textbf{Pencil Project} - zur Erstellung von GUI-Entwürfen
		\item \textbf{ArgoUML} - zur Erstellung von UML Diagrammen
		\item \textbf{Dia} - zur Erstellung weiterer Diagramme
	\end{itemize}
	\item Qualitätssicherung
	\begin{itemize}
		\item \textbf{JUnit} - zum Testen des Java Quellcodes
		\item \textbf{JaCoCo} (Java Code Coverage Library) - zur Analyse der Testabdeckung
		\item \textbf{Checkstyle} - Code-Analyse zur Prüfung des Programmierstils
	\end{itemize}
	\item Teamkommunikation
	\begin{itemize}
		\item \textbf{Google Groups} - als Mailingliste
		\item \textbf{Google Hangout} - für Sprach- und Videochatkonferenzen
	\end{itemize}
\end{itemize}

\section{Zeit- und Ressourcenplanung}

\subsection{Die Phasen des Projekts}
Die zeitliche Aufteilung dieses Softwareprojekts richtet sich nach den fünf Phasen des Wasserfallmodells mit der folgenden zeitlichen Aufteilung und Phasenverantwortlichen.\\
Die einzelnen Phasenverantwortlichen sind dabei dafür zuständig, federführend die ihnen zugeordnete Phase zu organisieren. Außerdem hat der Phasenverantwortliche die Aufgabe in dem Kolloquium am Ende seiner Phase diese in einem kleinen Vortrag vorzustellen und zu berichten was in dieser Zeit getan wurde und wie die Phase verlaufen ist.\\

\begin{tabular}[h]{lll}
	\hline
	\textbf{Phase} & \textbf{Phasendauer} & \textbf{Phasenverantwortlicher} \\
	\hline
	Pflichtenheft & 3 Wochen & Nick Vlasoff \\
	Entwurf & 4 Wochen & Philipp Löwer \\
	Implementierung & 4 (+ 2) Wochen & Anton Mehlmann und Enes Ördek \\
	Validierung & 3 Wochen & Simon Schürg \\
	Interne Abnahme und Abschlusspräsentation & 2 Wochen & Manuel Olk \\
	\hline
\end{tabular}

\subsection{Zeitliche Einteilung der einzelnen Module}
Da dieses Softwareprojekt im Rahmen der Lehrveranstaltungen \textit{Praxis der Softwareentwicklung (PSE)} und \textit{Teamarbeit in der Softwareentwicklung (TSE)} durchgeführt wird, muss sich der Arbeitsaufwand nach den ECTS-Punkten dieser Lehrveranstaltungen richten.\\\\
$PSE + TSE = 6 ECTS + 2 ECTS = 8 ECTS$\\
Ein ECTS-Punkt entspricht üblicherweise 30 Arbeitsstunden $\Rightarrow 8 * 30 Stunden = 240$ Stunden Arbeitsaufwand pro Person. Wir sind insgesamt 6 Personen d.h. es stehen uns $6 * 240 = 1440$ Personenstunden zur Verfügung die Aufgeteilt werden können.\\
Für die Phasen Entwurf und Implementierung planen wir insgesamt 720 Personenstunden ein. Eine genauere Aufteilung dieser Zeit auf die einzelnen Module des dieser Software ist in der folgenden Tabelle dargestellt.\\

\begin{tabular}[h]{lll}
	\hline
	\textbf{Modul} & \textbf{geschätzte Zeit} & \textbf{Verantwortlicher} \\
	\hline
	Import-Export-Modul & ? Stunden &  \\
	GUI-Design & ? Stunden &  \\
	evtl. GUI-Navigation und Oberflächenlogik & ? Stunden &  \\
	Algorithmen zur Berechnung der Mandatsverteilung & ? Stunden &  \\
	Algorithmen zum finden Paradoxer Situationen  & ? Stunden &  \\
	Diagramme und Kartographische Ansicht  & ? Stunden &  \\
	interne Datenhaltung  & ? Stunden &  \\
	Vergleichsansicht & ? Stunden &  \\
	\hline
	Summe & 720 Stunden & --------- \\
	\hline
\end{tabular}

\subsection{Ressourcen}
Jedes Teammitglied benötigt einen Personal Computer mit Leistungswerten über den Mindestanforderungen für die Entwicklung dieser Software.


\section{Ergänzungen}
\textit{ToDo: was gibt es sonst noch zu diesem Softwareprojekt zu sagen, was nicht in den anderen Punkten abgedeckt ist?}

\section{Glossar}
\begin{itemize}
	\item Listenplatz
	\item paradoxe Situationen
	\item Mandat	
	\item Direktmandat	
	\item Überhangmandat
	\item Ausgleichsmandat
\end{itemize}

\end{document}
