\documentclass[12pt,a4paper,titlepage]{article}
\usepackage[utf8]{inputenc}
\usepackage[german]{babel}
\usepackage{amsmath}
\usepackage{amsfonts}
\usepackage{amssymb}
\usepackage{setspace}
\usepackage{graphicx} %Um Bilder anzeigen zu können
\usepackage[top=1in, bottom=1.5in, left=1in, right=1in]{geometry}
\usepackage{endnotes}
\usepackage[section]{placeins}
\usepackage{fancyhdr}
\usepackage{hyperref}

\newcommand{\myma}{\fontfamily{pcr}\selectfont \textbf}
\newcommand{\mymo}{\fontfamily{pcr}\selectfont \textit}
\setlength{\parindent}{0pt}
\let\footnote=\endnote

\begin{document}
\pagestyle{fancy}

\begin{titlepage}
\vspace*{50pt}
\begin{center}
{\Huge Validierung\\[1cm] {\bfseries Praxis der Softwareentwicklung}\\[2cm] Entwicklung einer Software zur Berechnung der Mandatsverteilung im Deutschen Bundestag\\[1cm]Gruppe 1} \\
\vspace*{15pt}
{\normalsize Philipp Löwer, Anton Mehlmann, Manuel Olk, Enes Ördek, \\Simon Schürg, Nick Vlasoff}
\end{center}
\date{}

\vspace*{30pt}
\begin{figure}[h]
\centering
		\includegraphics[scale=0.6]{KIT-Logo.png}\\
		\vspace*{10pt}
		\Huge WS 2013 / 14
\end{figure}
\end{titlepage}
\newpage\thispagestyle{empty}\hspace{1em}\newpage
\def\Vhrulefill{\leavevmode\leaders\hrule height 0.7ex depth \dimexpr0.4pt-0.7ex\hfill\kern0pt}
\cfoot{{\Vhrulefill~  Seite \thepage   ~\Vhrulefill} \newline {\scriptsize KIT – Universität des Landes Baden-Württemberg und nationales Forschungszentrum in der Helmholtz-Gemeinschaft}}

\pagenumbering{roman} 

 
\newpage
\begin{onehalfspace}
\tableofcontents
\end{onehalfspace}
\newpage

\pagenumbering{arabic} 

\section{Einleitung}

Dieses Dokument ist im Zuge der Validierungsphase entstanden.

\subsection{Notationshinweise}
{\myma{Klassennamen}} werden in diesem Dokument textuell hervorgehoben, indem sie \textbf{fett} und in einer anderen Schriftart geschrieben werden.\newline
{\mymo{Methodennamen}} werden hervorgehoben, indem sie  \textit{kursiv} und ebenfalls in einer anderen Schriftart geschrieben werden.\newline
Außerdem wird Bundestagswahl im gesamten Entwurfsdokument durch BTW abgekürzt.
\newpage

\section{Globale Testfälle und Szenarien}
Im folgenden werden die im Pflichtenheft genannten Testszenarien durchgeführt und analysiert.

\begin{description}
	\item[\textbf{/T0010/}] Zwei Wahlen miteinander vergleichen: \\
	Das Vergleichsfenster öffnet wie vorgesehen, aber es gibt noch einige Ungereimtheiten.
	
	\begin{itemize}
	\item Wahl kann mit sich selbst verglichen werden, was keinen Sinn macht, wenn man in der Vergleichsansicht keine Stimmen ändern kann
	\item Wahlfenster öffnet sich im Hintergrund
	\item Berichtsfenster können nicht angezeigt werden
	\end{itemize}
	
	\item[\textbf{/T0020/}] Manuell einen negativen Wert als Stimmenanzahl eintragen: \\
	Exception wird geworfen, aber noch keine Fehlermeldung ausgegeben.
	\item[\textbf{/T0030/}] Manuell einen Buchstaben als Stimmenanzahl eintragen: \
	Folgende Meldungen werden ausgegeben.
	\begin{itemize}
	\item Nur positive ganze Zahlen erlaubt.
	\item Stimme konnte nicht geändert werden.
	\end{itemize}
	Daraufhin wird die Stimme wieder zurückgesetzt
	\item[\textbf{/T0040/}] Eine Fließkommazahl als Stimmenanzahl eintragen: \\
	siehe \textbf{/T0030/}
	\item[\textbf{/T0050/}] Erststimme in der Wahlkreisansicht verändern: \\
	funktioniert.
	\item[\textbf{/T0060/}] Die Funktion ``Diagramm wechseln'' testen:\\
	Funktion nicht in aktuellem Programm realisiert.
	\item[\textbf{/T0070/}] Die Funktion ``Rückgängig machen'' testen: \\
	Zurücksetzen funktioniert noch nicht.
	\item[\textbf{/T0080/}] Die Funktion ```Wiederherstellen'' testen: \\
Da rückgängig machen noch nicht funktioniert, kann wiederherstellen nicht angewählt werden.
\end{description}
\subsubsection{Import-/Exportverhalten}
Die folgenden Testfälle testen das Import-/Exportverhalten des Programms. Dabei wird vorausgesetzt, dass das Programm gestartet wurde und sich im Startzustand befindet. 
\begin{description}
	\item[\textbf{/T0110/}] Struktur einer Importdatei verändern: \newline
	Test: Gebiet gelöscht \newline
	Exception in Crawler ("Kein geeigneter Crawler gefunden") wurde geworfen, aber keine Fehlermeldung an Benutzer ausgegeben
	\item[\textbf{/T0120/}] 
	?? 
	\item[\textbf{/T0130/}] 
	Test: SPD zu CDU \\
	Keine Exception und keine Fehlermeldung. Programm funktioniert wie gewöhnlich
	\item[\textbf{/T0140/}] Nur eine Partei befindet sich in der Importdatei:\\
	Test: Alle Parteien außer CDU entfernt, Bundesländer und Wahlkreise für die Übersichtlichkeit entfernt, Stimmzahlen angepasst(z.B. Gesamtzweitstimmenanzahl in D) \\
	Endlosschleife 
	 \newline

	\item[\textbf{/T0150/}] Importdatei mit fehlerhaften Bundesländernamen:
	Test: Hamburg -> Hambe\\
	Exception im Mandatsrechner
	\item[\textbf{/T0160/}] Eigenen Wahlausgang erstellen: \\
	funktioniert.
\end{description}
\subsubsection{Korrekte Berechnung der Sitzverteilung}
Die folgenden Testfälle testen die korrekte Berechnung der Sitzverteilung. Dabei wird vorausgesetzt, dass  das Programm gestartet wurde und erfolgreich eine Importdatei geladen wurde.
\begin{description}
	\item[\textbf{/T0210/}] Ein Direktmandat fehlt: \newline
	\begin{itemize}
	\item Es kann immer nur eine Erststimmenanzahl geändert werden, dann muss berechnet werden
	\item Es können alle Erststimmenanzahlen auf 0 gesetzt werden, aber Direktmandat im Bundesland wird immer noch angezeigt
	\item Berechne- Knopf erscheint immer über Diagramm
	\item Exception: java.lang.OutOfMemoryError: Java heap space
	\end{itemize}
	
	\item[\textbf{/T0220/}] Mehrere Wahlkandidaten haben gleich viele Stimme in einem Wahlkreis: \newline
	\begin{itemize}
	\item Es wird kein Hinweis hinsichtlich einer Auslosung ausgegeben
	\item bleibt zu testen, ob überhaupt gelost wird
	\end{itemize}
	\item[\textbf{/T0230/}] Ein negatives Stimmgewicht in einer Wahl provozieren:
	fällt im Moment weg.
	\item[\textbf{/T0240/}] Partei mit drei Direktmandate und 2.9 Prozent der Zweitstimmen: \newline
	\begin{itemize}
	\item Bei 3 Direktmandaten passiert nichts
	\item Wahlgenerierung und Partei 3 Prozent der Zweitstimmen gegeben, sie war aber nicht im Bundestag vertreten
	\end{itemize}
	
	\item[\textbf{/T0250/}] Überhangmandat testen: \newline
\begin{itemize}
\item Überhangmandate werden in Landesansicht nicht angezeigt
\item Hinweis meiner Meinung nach überflüssig
\end{itemize}
	\item[\textbf{/T0260/}] Ausgleichsmandat testen: \newline
\begin{itemize}
\item Hinweis meiner Meinung nach nicht wichtig

	
\end{itemize}

\item[\textbf {weitere}]:
\begin{itemize}
\item leere Bewerberliste 
NullPointer in Mandatsrechner
\end{itemize}

\end{description}

\subsection{GUI-Test-Plan}
Folgender Ablauf sollte alles testen, was ein Benutzer mit unserem Programm machen kann:\\
Als aller erstes sollten alle kleinen Dinge getestet werden, dies sind:\\
 - Handbuch \\
 - About \\
 - Lizenz \\ 
Als nächstes sollte man die Bundestagswahl 2013 schließen und eine neue Wahl importieren. In dieser neuen Wahl kann man durch verschiedene Bundesländer und Wahlkreise hin und her wechseln. Anschließend sollte man in einem der Wahlkreise
jeweils eine Erst- und eine Zweitstimme ändern. Nun sollte der 'Berechne' Knopf betätigt werden. \\
Änderung einer weiteren Stimme und betätigen der 'Rückgängig'-Funktion erprobt die Funktion, eine Eingabe zu ändern. Hier kann auch als Eingabe ein Buchstabe, eine negative Zahl oder eine Zahl, die über die Anzahl der Wahlberechtigten geht, probiert werden. \\
Als nächstes kann man den 'Bericht' aufrufen und wieder schließen. \\
Zum Schluss sollte man die veränderte Wahl exportieren und wiederum importieren, um zu testen, ob der Import veränderter Wahlen funktioniert.\\
Nun kann man sich eine zufällige Wahl generieren lassen, deren Auswertung überprüft werden sollte. \\
Diese kann man dann mit der vorher importierten Wahl vergleichen. \\
Nachdem das Vergleichsfenster wieder geschlossen wurde, sollte man sich eine Wahl simulieren lassen, die das negative Stimmgewicht aufweist. \\
Zum Schluss sollte das Programm durch 'Schließen' beendet werden. \\
Dieser Plan umfasst alle Funktionen des Programmes. \\

\subsubsection{Fehler der GUI}
\begin{itemize}
\item Generiere-Knopf blieb aktiv, nachdem man den Namen löschte \\
Das Problem bei diesem Fehler war, dass ein ActionListener verwendtet wurden. Aus diesem Grund wurde die
Abfrage, ob der aktuelle Name gesetzt ist, nur abgefragt, sobald der Benutzer den Enter-Knopf betätigte.
Durch den neuen KeyListener wird bei kleinster Änderung abgefragt, ob sich noch Text in dem JTextField befindet. \\

\item Berichtsknopf \\
Bei der Präsentation der Implementierungsphase ergab sich, dass die Einführung eines Knopfes der den Bericht erscheinen
lässt handlicher wäre. Vorher musste man auf das Diagramm klicken, was nicht gerade intuitiv war. \\
Wegen dem neuen Button waren Änderungen am Layout von Nöten. Die Klasse DiagrammFenster enthält jetzt das GridBagLayout,
wobei unter dem Diagramm der Knopf angezeigt wird. \\

\item Änderungen von Erst- und Zweitstimmen \\
Nach der Implementierung war es noch möglich negative Stimmwerte einzutragen. Dies wurde behoben, indem eine einfache Abfrage eingeführt wurde, sobald der Wert unter 0 ist wird eine NumberFormatException geworfen. \\
Außerdem war es möglich mehr Stimmen abzugeben als es Wahlberechtigte im Land gab. Auch dies wurde durch eine Extra-Abfrage behoben. \\
Folgender Test wurde ausgeführt, um die Korrektheit der Stimmenänderung zu testen:\\
- Nach .csv-Datei sind im Wahlkreis Karlsruhe-Stadt noch 57624 Wahlberechtigte übrig. Nun ändern wir die Erststimmen einer Partei, die 0 bekommen hat, auf 57625. Wie erwartet, wird angezeigt, dass nur noch 57624 Wahlberechtigte übrig sind. \\
- Eine Änderung um den Wert 57624 ist ohne Probleme möglich.\\
- Ein ähnlicher Test wurde auch für die Zweitstimmen erprobt.\\

\item Tabwechsel
Das Wechseln der Tabs enthielt den Fehler, dass die Bundestagswahl, die die Steuerung hält, nicht zu der geändert wurde, die zum neuen Tab gehört.\\
Dies wurde korrigiert, indem der TabLeiste zu jedem Tab ein MouseListener hinzugefügt wurde.\\

\end{itemize}
	
\section{Vorschau auf die nächste Phase}


\begingroup
\parindent 0pt
\parskip 2ex
\def\enotesize{\normalsize}

\endgroup
\end{document}
