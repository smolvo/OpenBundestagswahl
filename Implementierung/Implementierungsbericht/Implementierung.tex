\documentclass[12pt,a4paper,titlepage]{article}
\usepackage[utf8]{inputenc}
\usepackage[german]{babel}
\usepackage{amsmath}
\usepackage{amsfonts}
\usepackage{amssymb}
\usepackage{setspace}
\usepackage{graphicx} %Um Bilder anzeigen zu können
\usepackage[top=1in, bottom=1.5in, left=1in, right=1in]{geometry}
\usepackage{endnotes}
\usepackage[section]{placeins}
\usepackage{fancyhdr}
\usepackage{hyperref}

\newcommand{\myma}{\fontfamily{pcr}\selectfont \textbf}
\newcommand{\mymo}{\fontfamily{pcr}\selectfont \textit}
\setlength{\parindent}{0pt}
\let\footnote=\endnote

\begin{document}
\pagestyle{fancy}

\begin{titlepage}
\vspace*{50pt}
\begin{center}
{\Huge Implementierung\\[1cm] {\bfseries Praxis der Softwareentwicklung}\\[2cm] Entwicklung einer Software zur Berechnung der Mandatsverteilung im Deutschen Bundestag\\[1cm]Gruppe 1} \\
\vspace*{15pt}
{\normalsize Philipp Löwer, Anton Mehlmann, Manuel Olk, Enes Ördek, \\Simon Schürg, Nick Vlasoff}
\end{center}
\date{}

\vspace*{30pt}
\begin{figure}[h]
\centering
		\includegraphics[scale=0.6]{KIT-Logo.png}\\
		\vspace*{10pt}
		\Huge WS 2013 / 14
\end{figure}
\end{titlepage}
\newpage\thispagestyle{empty}\hspace{1em}\newpage
\def\Vhrulefill{\leavevmode\leaders\hrule height 0.7ex depth \dimexpr0.4pt-0.7ex\hfill\kern0pt}
\cfoot{{\Vhrulefill~  Seite \thepage   ~\Vhrulefill} \newline {\scriptsize KIT – Universität des Landes Baden-Württemberg und nationales Forschungszentrum in der Helmholtz-Gemeinschaft}}

\pagenumbering{roman} 

 
\newpage
\begin{onehalfspace}
\tableofcontents
\end{onehalfspace}
\newpage

\pagenumbering{arabic} 

\section{Einleitung}
Die dritte Phase unseres Projektes - die Implementierung. Unser Ziel in dieser Phase ist es, die bisherigen Errungenschaften in der Pflichten- und Entwurfsphase als ausführbares Programm umzusetzen und dabei möglichst wenig von den bisherigen Entwürfen auszuweichen. Hierbei ist uns jedoch aufgefallen, dass Veränderungen am Entwurf unumgänglich sind. Der Grund hierfür ist, dass bestimmte Sachen einfach nicht beachtet wurden, und somit übersehen worden sind. \\
Dieses Dokument wird das fertige Programm mit seinen Funktionen erläutern und alle Veränderungen mit den zugehörigen Entwurfsentscheidungen ausführlich erklären. Anschließend werden wir eine Einblick in die nächste Phase geben. \\

Dieses Dokument ist im Zuge der Implementierungsphase entstanden. Ziel dieser Phase, ist die Umsetzung der in der Pflichten- und Entwurfsphase festgelegten Strukturen und Prozessabläufe unter Berücksichtigung gegebener Rahmenbedingungen, Regeln und Zielvorgaben. \\
Da sich jedoch während der Implementierung Sachverhalte ergeben, die mit dem eigentlichen Entwurf nicht vollständig zu vereinbaren sind, ist es notwendig einige Änderungen bzw. Anpassungen, aber auch Erweiterungen vorzunehmen. Diese, vom eigentlichen Plan abweichenden Entscheidungen, werden im Folgenden erläutert. \\
Abschließend wird ein kurzer Ausblick auf die nächste Phase gegeben. \\


\subsection{Notationshinweise}
{\myma{Klassennamen}} werden in diesem Dokument textuell hervorgehoben, indem sie \textbf{fett} und in einer anderen Schriftart geschrieben werden.\newline
{\mymo{Methodennamen}} werden hervorgehoben, indem sie  \textit{kursiv} und ebenfalls in einer anderen Schriftart geschrieben werden.\newline
Außerdem wird Bundestagswahl im gesamten Entwurfsdokument durch BTW abgekürzt.
\newpage

\section{Pakete}

\subsection{Datenmodell}
Da die verwendeten Listen im Datenmodell durch die Berechnungen und Zuweisungen recht groß wurden, wurde zusätzliche Funktionalität in das Datenmodell gebracht. Dies erleichterte den Zugriff auf die benötigten Daten für verschiedene Komponente wie zum Beispiel {\myma{Mandatsrechner}}, {\myma{Wahlgenerator}} und {\myma{GUI}}.
Zudem wurde die Klasse {\myma{BerichtDaten}} für die Klasse {\myma{Sitzverteilung}} erstellt. Dies war notwendig, damit die Berichttabelle in der GUI korrekt befüllt werden kann. Dabei hält die Klasse {\myma{BerichtDaten}} fünf Listen die jeweils die dazugehörigen Spalten befüllen. 
Kandidaten haben nun einen Namen und einen bestimmten Platz in der Landesliste.
\subsection{Import/Export}	
Die Import-Export-Komponente wurde im Laufe der Implementierung stark angepasst. Anders als im Entwurf, haben wir das Exportieren vom Importieren getrennt. Da die Namen und der feste Platz in der Ladenliste mitgespeichert werden, muss eine zusätzliche .csv-Datei importiert werden. Mithilfe dieser Wahlbewerber-Datei werden die vorher ausgelesenen Kandidaten befüllt. Die Wahlbewerber-Datei für die Bundestagswahl 2013 wird im Programm mit übergeben und kann im Notfall für Bundestagswahl 2009 genutzt werden. Zudem wurde eine Config-Datei(ebenfalls im .csv-Format) hinzugefügt, die die Einwohnerzahl der Bundesländer und die Farben der Partei beinhaltet. Dadurch müssen diese Werte nicht mehr im Programmcode gespeichert und können durch das Editieren der Datei einfach angepasst werden. 

\subsection{GUI und GUI-Logik}
3 Ansichten 
\subsubsection{Programmfenster}
Das {\myma{Programmfenster}} ist der eigentliche Eintrittspunkt in das Programm, d.h. es enthält die Main- Klasse und wird beim Start als Erstes ausgeführt. Dies bietet sich an, da das {\myma{Programmfenster}} das Erste sein soll, was der Benutzer sieht, da er damit ja interagieren muss. \\
Wie bereits im Entwurf festgehalten, enthält das {\myma{Programmfenster}} eine Liste von {\myma{Wahlfenstern}}. Diese werden mithilfe einer {\myma{TabLeiste}}, die ebenfalls vom {\myma{Programmfenster}} gehalten wird, realisiert. \\
Zusätzlich besitzt es ein {\myma{Menu}}, welches dem Benutzer ermöglicht, den gewünschten Befehl auszuwählen und ausführen zu lassen, ohne genaue Steuerbefehle kennen und anwenden zu müssen. \\
	
\subsection{Mandatsrechner}
Die Klasse {\myma{Mandatsrechner2009}} berechnet die Sitzverteilung nach Sainte-Lagu\"e/Schepers ohne Ausgleichsmandate und {\myma{Mandatsrechner2013}} berechnet die Sitzverteilung ebenfalls nach Sainte-Lagu\"e/Scheper, aber mit Ausgleichsmandate. Da der \\{\myma{Mandatsrechner2013}} dadurch den {\myma{Mandatsrechner2009}} zur Berechnung nutzt, fällt die Notwendigkeit der Oberklasse {\myma{Mandatsrechner}} weg. Deswegen hält der \\ {\myma{Mandatsrechner2013}} ein Objekt der Klasse {\myma{Mandatsrecher2009}}. Zudem wurde noch in {\myma{Mandatsrechner2009}} das Verteilungsverfahren nach d'Hondt implementiert, damit eine alternative Berechnung der Sitzverteilung möglich ist. Die Überladung der Methoden {\mymo{bechne(Gebiet gebiet)}} wurde aufgehoben, da die Berechnung nicht nach Gebieten sondern nach Ober- und Unterverteilung orientiert ist. Damit der {\myma{Mandatsrechner2013}} möglichst viel wieder verwendet werden kann, wurden Bereiche die in beiden Berechnungsklassen Verwundung finden ausgelagert. Für die Implementierung des Entwurfsmuster Einzelstück wurden möglichst wenig globale Variablen, die vor jeder Berechnung neu initialisiert werden, verwendet, damit in dem Mandatsrechner nicht gewollte Zustände ausgeschlossen werden können.
	
	\subsection{Wahlgenerator}
	
	\subsection{Wahlvergleich}
	
	\subsection{Steuerung}
	
	\subsection{Sonstiges}

\section{Vorschau auf die nächste Phase}

	\subsection{Ideen und Ziele}

	\subsection{Zeitplan}
	
\begingroup
\parindent 0pt
\parskip 2ex
\def\enotesize{\normalsize}

\endgroup
\end{document}
